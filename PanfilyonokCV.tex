%%%%%%%%%%%%%%%%%%%%%%%%%%%%%%%%%%%%%%%%%
% Medium Length Professional CV
% LaTeX Template
% Version 2.0 (8/5/13)
%
% This template has been downloaded from:
% http://www.LaTeXTemplates.com
%
% Original author:
% Trey Hunner (http://www.treyhunner.com/)
%
% Important note:
% This template requires the resume.cls file to be in the same directory as the
% .tex file. The resume.cls file provides the resume style used for structuring the
% document.
%
%%%%%%%%%%%%%%%%%%%%%%%%%%%%%%%%%%%%%%%%%

%----------------------------------------------------------------------------------------
%	PACKAGES AND OTHER DOCUMENT CONFIGURATIONS
%----------------------------------------------------------------------------------------

\documentclass{resume} % Use the custom resume.cls style

\usepackage[left=0.75in,top=0.6in,right=0.75in,bottom=0.6in]{geometry} % Document margins
\usepackage[T2A]{fontenc}
\usepackage[utf8]{inputenc}
\usepackage[russian]{babel}
\usepackage{array}
\usepackage{hyperref}

\newcommand{\Fsh}{F{\newcommand{\lserif{}}\#}}
\newcommand{\Csh}{C{\newcommand{\lserif{}}\#}}
\newcommand{\tab}[1]{\hspace{.2667\textwidth}\rlap{#1}}
\newcommand{\itab}[1]{\hspace{0em}\rlap{#1}}

\name{Dmitriy Panfilyonok} % Your name
\address{Saint-Petersburg} % Your address
\address{dmitriy.panfilyonok@gmail.com \\ \href{https://github.com/dpanfilyonok}{github.com/dpanfilyonok}} 

\begin{document}

%----------------------------------------------------------------------------------------
%	EDUCATION SECTION
%----------------------------------------------------------------------------------------

\begin{rSection}{Education}

{\bf Saint-Petersburg State University} \hfill {2018 --- 2022} \\
{\em Software and Administration of Information Systems} \\
{\em Faculty of Math and Mechanics} \hfill {\em bachelor's degree} 

\end{rSection}

%----------------------------------------------------------------------------------------
%	TECHNICAL STRENGTHS SECTION
%----------------------------------------------------------------------------------------

\begin{rSection}{Skills and Competencies}

\begin{tabular}{ @{} >{\bfseries}l @{\hspace{6ex}} l }
Programming languages & \Csh, \Fsh, Scala \\
Frameworks & ASP.NET Core, EF Core, OpenCL \\
Tools & Git, PostgreSQL, Docker \\
Languages & English B2, Russian Native
\end{tabular}

\end{rSection}

%----------------------------------------------------------------------------------------
%	WORK EXPERIENCE SECTION
%----------------------------------------------------------------------------------------

\begin{rSection}{Work Experience}

\begin{rSubsection}{JetBrains Research}{November 2020 --- February 2022}{Researcher, Programming Languages and Tools lab}{}
\item Worked on the implementation of the GraphBLAS API for the \Fsh~programming language.
\item Implemented support for generalized atomic operations in a high-performance OpenCL library using the \Fsh~programming language.
\item \textbf{Stack}: \Fsh, OpenCL, GPU, metaprogramming, transpilation, sparse structures.
\end{rSubsection}

\end{rSection}

%----------------------------------------------------------------------------------------
%	PROJECTS
%----------------------------------------------------------------------------------------

\begin{rSection}{Projects}

\begin{rSubsection}{\href{https://github.com/dpanfilyonok/Brahma.FSharp.REPL}{A web application that allows to translate \Fsh~code into OpenCL}}{2022}{Pet-project}{}
\item Implemented the server side of the application using ASP.NET Core and Giraffe. 
\item Implemented the client side of the application using React.
\item Configured the deployment environment in Docker.
\item Deployed the app to Heroku and Github Pages.
\end{rSubsection}

\begin{rSubsection}{\href{https://github.com/dpanfilyonok/Brahma.FSharp}{Improving the GPGPU generalized computation library on \Fsh}}{2021}{Graduate work}{}
\item Improved the translator that converts AST of the \Fsh~language into OpenCL code.
\item Improved OpenCL device memory management model.
\item Improved API for parallel execution of OpenCL kernels.
\end{rSubsection}

\begin{rSubsection}{\href{https://github.com/dpanfilyonok/GraphBLAS-sharp}{Development of linear algebra based graph processing library on \Fsh}}{2020}{}{}
\item Implemented a subset of operations on sparse matrices on GPU using OpenCL.
\item Implemented some graph processing algorithms in terms of linear algebra in \Fsh.
\item Set up property-based testing with FsCheck and Expecto, as well as CI with GitHub Actions and AppVeyor.
\end{rSubsection}

\end{rSection}

\begin{rSection}{Courses}
\itab{Software Design}  \tab{}  \tab{Databases and DBMS} \\
\itab{Functional programming} \tab{} \tab{Computer networks and data storage} \\
\itab{Introduction to Linux} \tab{}  \tab{Distributed information processing}
\end{rSection}

\end{document}

