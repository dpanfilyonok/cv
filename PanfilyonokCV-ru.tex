%%%%%%%%%%%%%%%%%%%%%%%%%%%%%%%%%%%%%%%%%
% Medium Length Professional CV
% LaTeX Template
% Version 2.0 (8/5/13)
%
% This template has been downloaded from:
% http://www.LaTeXTemplates.com
%
% Original author:
% Trey Hunner (http://www.treyhunner.com/)
%
% Important note:
% This template requires the resume.cls file to be in the same directory as the
% .tex file. The resume.cls file provides the resume style used for structuring the
% document.
%
%%%%%%%%%%%%%%%%%%%%%%%%%%%%%%%%%%%%%%%%%

%----------------------------------------------------------------------------------------
%	PACKAGES AND OTHER DOCUMENT CONFIGURATIONS
%----------------------------------------------------------------------------------------

\documentclass{resume} % Use the custom resume.cls style

\usepackage[left=0.75in,top=0.6in,right=0.75in,bottom=0.6in]{geometry} % Document margins
\usepackage[T2A]{fontenc}
\usepackage[utf8]{inputenc}
\usepackage[russian]{babel}
\usepackage{array}
\usepackage{hyperref}

\newcommand{\Fsh}{F{\newcommand{\lserif{}}\#}}
\newcommand{\Csh}{C{\newcommand{\lserif{}}\#}}
\newcommand{\tab}[1]{\hspace{.2667\textwidth}\rlap{#1}}
\newcommand{\itab}[1]{\hspace{0em}\rlap{#1}}

\name{Дмитрий Панфилёнок} % Your name
\address{Санкт-Петербург} % Your address
\address{dmitriy.panfilyonok@gmail.com \\ \href{https://github.com/dpanfilyonok}{github.com/dpanfilyonok}} 

\begin{document}

%----------------------------------------------------------------------------------------
%	EDUCATION SECTION
%----------------------------------------------------------------------------------------

\begin{rSection}{Образование}

{\bf Санкт-Петербургский Государственный Университет} \hfill {2018 --- 2022} \\
{\em Математическое обеспечение и администрирование информационных систем} \\
{\em Математико-механический факультет} \hfill {\em бакалавриат} 

\end{rSection}

%----------------------------------------------------------------------------------------
%	TECHNICAL STRENGTHS SECTION
%----------------------------------------------------------------------------------------

\begin{rSection}{Умения и Навыки}

\begin{tabular}{ @{} >{\bfseries}l @{\hspace{6ex}} l }
Языки программирования & \Csh, \Fsh, Scala \\
Фреймворки & ASP.NET Core, EF Core, OpenCL \\
Инструменты & Git, PostgreSQL, Docker \\
Иностранные языки & Английский B2
\end{tabular}

\end{rSection}

%----------------------------------------------------------------------------------------
%	WORK EXPERIENCE SECTION
%----------------------------------------------------------------------------------------

\begin{rSection}{Опыт}

\begin{rSubsection}{JetBrains Research}{Ноябрь 2020 --- Февраль 2022}{Исследователь, Лаборатория языковых инструментов}{}
\item Работал над реализацией GraphBLAS API для языка программирования \Fsh.
\item Реализовал поддержку обобщенных атомарных операций в библиотеке для высокопроизводительных вычислений на OpenCL с использованием языка программирования \Fsh.
\item \textbf{Стек}: \Fsh, OpenCL, GPU, метапрограммирование, транспиляция, разреженные структуры.
\end{rSubsection}

\end{rSection}

%----------------------------------------------------------------------------------------
%	PROJECTS
%----------------------------------------------------------------------------------------

\begin{rSection}{Проекты}

\begin{rSubsection}{\href{https://github.com/dpanfilyonok/Brahma.FSharp.REPL}{Веб-приложение, позволяющее транслировать код на \Fsh~в OpenCL}}{2022}{Pet-проект}{}
\item Реализовал серверную часть приложения с помощью ASP.NET Core и Giraffe. 
\item Реализовал клиентскую часть приложения с помощью React.
\item Настроил среду развертывания в Docker.
\item Развернул приложение на Heroku и Github Pages.
\end{rSubsection}

\begin{rSubsection}{\href{https://github.com/dpanfilyonok/Brahma.FSharp}{Улучшение библиотеки обобщенных вычислений на GPGPU на \Fsh}}{2021}{Дипломная работа}{}
\item Улучшил транслятор, преобразующий AST языка \Fsh~в код на OpenCL.
\item Улучшил модель управления памятью OpenCL устройства.
\item Улучшил API для параллельного исполнения ядер OpenCL.
\end{rSubsection}

\begin{rSubsection}{\href{https://github.com/dpanfilyonok/GraphBLAS-sharp}{Разработка библиотеки обработки графов в терминах линейной алгебры на \Fsh}}{2020}{}{}
\item Реализовал подмножество операций над разреженными матрицами на GPU с помощью OpenCL.
\item Реализовал некоторые алгоритмы обработки графов в терминах линейной алгебры на \Fsh.
\item Настроил property-based тестирование средствами FsCheck и Expecto, а также CI средствами GitHub Actions и AppVeyor.
\end{rSubsection}

\end{rSection}

\begin{rSection}{Курсы}
\itab{Проектирование программного обеспечения}  \tab{}  \tab{Базы данных и СУБД} \\
\itab{Функциональное программирование} \tab{} \tab{Компьютерные сети и хранилища данных} \\
\itab{Введение в Linux} \tab{}  \tab{Распределенная обработка информации}
\end{rSection}

\end{document}

