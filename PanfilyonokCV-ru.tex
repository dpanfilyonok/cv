%%%%%%%%%%%%%%%%%%%%%%%%%%%%%%%%%%%%%%%%%
% Medium Length Professional CV
% LaTeX Template
% Version 2.0 (8/5/13)
%
% This template has been downloaded from:
% http://www.LaTeXTemplates.com
%
% Original author:
% Trey Hunner (http://www.treyhunner.com/)
%
% Important note:
% This template requires the resume.cls file to be in the same directory as the
% .tex file. The resume.cls file provides the resume style used for structuring the
% document.
%
%%%%%%%%%%%%%%%%%%%%%%%%%%%%%%%%%%%%%%%%%

%----------------------------------------------------------------------------------------
%	PACKAGES AND OTHER DOCUMENT CONFIGURATIONS
%----------------------------------------------------------------------------------------

\documentclass{resume} % Use the custom resume.cls style

\usepackage[left=0.75in,top=0.6in,right=0.75in,bottom=0.6in]{geometry} % Document margins
\usepackage[T2A]{fontenc}
\usepackage[utf8]{inputenc}
\usepackage[russian]{babel}
\usepackage{array}

\newcommand{\Fsh}{F{\newcommand{\lserif{}}\#}\ }
\newcommand{\Csh}{C{\newcommand{\lserif{}}\#}\ }

\name{Дмитрий Панфилёнок} % Your name
\address{Санкт-Петербург} % Your address
\address{dmitriy.panfilyonok@gmail.com \\ github.com/dpanfilyonok} 

\begin{document}

%----------------------------------------------------------------------------------------
%	EDUCATION SECTION
%----------------------------------------------------------------------------------------

\begin{rSection}{Образование}

\begin{rSubsection}{Санкт-Петербургский Государственный Университет}{2018 --- сейчас}{Математическое обеспечение и администрирование информационных систем \\ Математико-механический факультет}{бакалавриат}
\end{rSubsection}

\end{rSection}

%----------------------------------------------------------------------------------------
%	WORK EXPERIENCE SECTION
%----------------------------------------------------------------------------------------

\begin{rSection}{Проекты}

\begin{rSubsection}{Улучшение библиотеки обобщенных вычислений на GPGPU на \Fsh}{2021}{Дипломная работа}{}
\item Улучшил транслятор, преобразующий AST языка \Fsh в код на OpenCL
\item Улучшил модель управления памятью OpenCL устройства
\item Улучшил API для параллельного исполнения ядер OpenCL
\end{rSubsection}

\begin{rSubsection}{Разработка библиотеки обработки графов в терминах линейной алгебры на \Fsh}{2020}{Курсовая работа}{}
\item Разработал реализацию GraphBLAS API для языка программирования \Fsh
\item Реализовал множество операций над разреженными матрицами на GPU с помощью OpenCL
\item Настроил property-based тестирование средствами FsCheck и Expecto, а также CI средствами GitHub Actions и AppVeyor
\end{rSubsection}

\end{rSection}

%----------------------------------------------------------------------------------------
%	TECHNICAL STRENGTHS SECTION
%----------------------------------------------------------------------------------------

\begin{rSection}{Умения и Навыки}

% \begin{rSubsection}{Языки программирования и технологии}{}{}{}
% \item \Fsh, \Csh, SQL, Git (средний уровень)
% \item OpenCL, Linux (есть опыт)
% \item C/C++, Python, R, x86, Android, TensorFlow (знакомство)
% \end{rSubsection}

% \begin{tabular}{l l l m{2cm}}
% & средний уровень & есть опыт & знакомство \\ 
% \textbf{Языки программирования} & F\# & C\# & C/C++, R, Python \\
% \textbf{Инструменты} & .NET, OpenCL, Git & Linux, Bash & x86, SAFE Stack, ASP.NET Core, Docker \\
% Иностранные языки & Английский, Немецкий
% \end{tabular}

\begin{tabular}{l l l}
& \textbf{Языки программирования} & \textbf{Инструменты} \\ 
\textbf{Средний уровень} & \Fsh & .NET, OpenCL, Git \\
\textbf{Есть опыт} & \Csh, Scala & Linux, Bash, SQL  \\
\textbf{Знакомство} & C/C++, R, Python & x86, Android, ASP.NET Core, Spark, Docker \\
\end{tabular}
\\
\end{rSection}

\begin{rSection}{Языки}
\begin{itemize}
\item Русский (родной)
\item Английский (B2, Upper Intermediate)
\end{itemize}
\end{rSection}

\end{document}

